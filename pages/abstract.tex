\chapter{\abstractname}

In modern computing environments, managing concurrent access to shared resources, such as database tables, file regions, or memory segments, is a significant challenge. 
Traditional synchronization methods often rely on a single lock, leading to inefficiencies and performance bottlenecks under high concurrency. 
Range locking offers a more refined solution by partitioning shared resources into multiple segments, allowing different processes to access these segments concurrently. 

Although range locking plays a crucial role in various systems, including database management systems (DBMS), file systems, and operating systems, it remains an under-studied topic. 
Existing implementations often suffer from contention and inefficiencies, particularly under high concurrency, highlighting the need for more scalable solutions.

This thesis introduces a new lock-free concurrent range-locking mechanism to overcome these challenges.
Our method improves upon previous designs by eliminating bottlenecks and ensuring high performance in heavily concurrent environments.
Our evaluation shows that the proposed method outperforms existing approaches. 
This thesis provides a comprehensive assessment of the development and evaluation of this new range lock, offering a comprehensive assessment of its effectiveness in modern computing systems.
