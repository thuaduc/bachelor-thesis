\chapter{\abstractname}

In modern computing environments, the management of concurrent access to shared resources, such as database tables, file regions, or memory segments, represents a significant challenge. 
Range locking provides a solution to this challenge by partitioning shared resources into arbitrarily-sized segments, thereby allowing different processes to access these segments concurrently.

Despite its crucial role in various systems, range locking remains an under-researched topic. 
Existing implementations often suffer from contention and inefficiencies, particularly in highly concurrent environments, underscoring the necessity for more scalable solutions.

In this thesis, we propose a new lock-free concurrent range-locking mechanism to overcome these challenges. 
Our method improves upon previous designs by eliminating bottlenecks and ensuring high performance in heavily concurrent environments. 
Our evaluation demonstrates that the proposed method outperforms existing approaches. 
This thesis provides an in-depth exploration of our method, offering a comprehensive assessment of its effectiveness in modern computing systems.