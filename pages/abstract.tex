\chapter{\abstractname}

Large-scale distributed systems and applications often face significant challenges in managing concurrent access to shared resources. For instance, high-performance computing environments frequently require efficient handling of numerous simultaneous access requests to various parts of data sets. Similarly, disaggregated memory systems must support multiple clients accessing the same memory space concurrently, often with diverse access patterns. In such contexts, it becomes crucial to coordinate and manage these concurrent accesses effectively to ensure correctness and performance. 

Concurrent range locks address these challenges by providing mechanisms for efficient and accurate management of overlapping and non-overlapping ranges of resources, thereby enabling scalable and reliable access control in complex, high-demand environments.

Previous studies have examined various range lock techniques, with the Linux kernel currently employing a range tree accompanied by a spin lock for managing access to ranges. This single spinlock, however, introduces bottlenecks. Song et al. proposed an enhancement by integrating a skip list with the spinlock, offering a more efficient and less heavyweight solution than traditional interval trees, though challenges with contention persist. Alternatively, Kogan et al. developed a lock-free range lock utilizing a concurrent linked list, each node representing an acquired range. This method addresses limitations found in existing range locks but at the cost of reduced efficiency in insertion and search operations compared to tree-based structures.

In the scope of this research, we propose a new lock-free concurrent range-locking design. We address previous bottleneck issues by leveraging a probabilistic concurrent skip list and removing the lock while maintaining high performance. The proposed mechanism will be developed and evaluated under heavy concurrent access, ensuring correctness in overlapping ranges and concurrent operations. Performance comparisons with existing state-of-the-art approaches will comprehensively assess its effectiveness.

