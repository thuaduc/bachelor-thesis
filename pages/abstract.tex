\chapter{\abstractname}

In modern computing environments, efficient locking mechanisms are vital for the performance of databases, file systems, and operating systems. Traditional single-lock techniques often lead to performance bottlenecks in high-concurrency scenarios. Range locks offer a refined approach by partitioning shared resources into multiple segments, each of which can be exclusively acquired by different processes, thus improving performance.

As database sizes grow, locking the entire database is impractical due to poor throughput and high latency. Existing key-range locking methods are complex and not suitable for general DBMS operations. Therefore, a new technique, such as range locks, is necessary.

Previous approaches, including the Linux kernel's range tree with an internal spinlock and skip lists with spinlocks, face contention issues. A lock-free range lock using a concurrent linked list has been proposed, but it suffers from slow insertion and lookup operations.

This research proposes a new concurrent range-locking design leveraging a probabilistic concurrent skip list with per-node locks, addressing previous bottlenecks and maintaining high performance. The proposed mechanism will be developed and evaluated under heavy concurrent access, ensuring correctness in overlapping ranges and concurrent operations. Performance comparisons with existing state-of-the-art approaches will provide a comprehensive assessment of its effectiveness.

