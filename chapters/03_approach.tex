% !TeX root = ../main.tex
% Add the above to each chapter to make compiling the PDF easier in some editors.

\chapter{Approach}\label{chapter:approach}

In this research's scope, we propose a new concurrent range lock design that leverages a probabilistic concurrent skip list\parencite{herlihy2006provably, herlihy2020art}. It consists of two main functions:
\begin{itemize}
    \item \textbf{try\_lock}: The \texttt{try\_lock} function searches for the required range \texttt{([start, start+len])} in the skip list. If an overlapping range exists, indicating another thread is modifying that range, the requesting thread must wait and retry. If not, the range is added to the list, signaling that the range is reserved.
    \item \textbf{release\_lock}: The \texttt{release\_lock} function releases the lock by finding the address range in the skip list and removing it accordingly.
\end{itemize} 

Our range lock design also utilizes the per-node lock instead of an interval lock, thus addressing the bottleneck problem of the spinlock-based range lock and maintaining the lock's high level of performance. 