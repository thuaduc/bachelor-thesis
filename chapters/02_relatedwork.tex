% !TeX root = ../main.tex
% Add the above to each chapter to make compiling the PDF easier in some editors.

\chapter{Related work}\label{chapter:relatedwork}

This chapter reviews the existing research on range locks, focusing on different aspects of the problem and the solutions researchers have proposed. The following sections provide a detailed examination of different implementations, highlighting their limitations. This analysis aims to show the evolution of range-locking mechanisms and the ongoing efforts to improve their performance and scalability.

\section{Range Lock in the Linux Kernel}

Jan Kara introduced a range-locking mechanism for the Linux kernel~\parencite{linuxRangeLockImpl2013}. The implementation leverages an interval tree to manage range locks and employs a spin lock for synchronization. Each lock is represented as a node in the tree.

\subsection*{How it works}

When a thread wants to acquire a range, it first acquires a spinlock. Then, it traverses the tree to determine the number of locks intersecting with the given range and find the. Next, the thread inserts a node describing its range into the tree and releases the spinlock. If the number of intersecting locks is zero, the thread can access and write to the critical section it acquires. Otherwise, it waits until those locks are released when the number of intersecting locks drops to zero. When a thread completes its range, it acquires the spin lock, removes its node from the tree, updates the number of overlapping locks, and frees the spinlock. It guarantees that each range is locked only after all previous conflicting range locks requested have been unlocked, thus achieving fairness and avoiding livelocks.
    
\subsection*{Drawbacks}

\textbf{Single Point of Contention.}
On this range lock, all operations relies on a single spinlock to protect the entire tree. Under heavy concurrent access, this easily becomes a contention point, limiting the system's performance.
    
\textbf{Limitation of FIFO Order.} 
Consider three exclusive lock requests for the ranges \( A = [1..3], \; B = [2..7], \; \text{and} \; C = [4..5] \), arriving in that order. While A holds the lock, B is blocked because it overlaps with A, and C is blocked behind B. However, in practice, C does not overlap with A and could proceed without the FIFO restriction. This unnecessary blocking reduces the overall efficiency and concurrency of the system.

\section{Skip List-Based Range Lock}

Song et al.~\parencite{song2013parallelizing} introduced a dynamic and fine-grained range-locking design to enhance the implementation of the Linux kernel. Their range lock utilizes a skip list~\parencite{pugh1990skip} to dynamically manage the address ranges that are currently locked.

\subsection*{How it works}

When a thread requests a specific range \texttt{[start, start+len)}, the range lock searches for it in the skip list. If an existing or overlapping range is found in the skip list, it means that another thread is currently modifying the specific range, and the requesting thread must wait and then retry. If no overlapping range is found, the range is added to the skip list, indicating that the range lock has been acquired. Releasing a range involves deleting the corresponding range from the skip list.

Compared to the interval tree, the skip list is more lightweight and efficient, and it can efficiently perform searches for overlapping ranges.

\subsection*{Drawbacks}

\textbf{Single Contention Point:} 
Similar to the one found in the Linux kernel, this range lock is also protected by a spinlock. Hence, the same bottleneck issue still need to be addressed  

\section{Concurrent Linked List-Base Range Lock}

Kogan et al.~\parencite{kogan2020scalable} introduced a novel range lock based on a concurrent linked list, where each node represents an acquired range. This design aims to provide a lock-free mechanism, addressing some critical shortcomings of previous range-locking implementations. In a lock-free system, processes can proceed without being blocked by locks held by other processes, thereby improving performance and scalability. 

\subsection*{How it works}

The proposed method involves inserting acquired ranges into a linked list sorted by their starting points, ensuring that only one range from a group of overlapping ranges can be inserted using an atomic \texttt{compare-and-swap} operation. A significant difference in this method compared to the previous one is that a node has two statuses: \texttt{marked} (logically deleted) or \texttt{unmarked} (present). 

When a thread wants to acquire a range, it iterates through the skip list. If it reaches a marked node, it simply removes it using CAS and continues to iterate. If the current node is protecting a range that overlaps, it simply waits until that node is deleted. Otherwise, a node is inserted into the list, signaling that the range is acquired. To release a range, the thread marks the node. 
\subsection*{Drawbacks}

\textbf{Linked List Inefficiency.}
This design implements a lock-free mechanism that effectively addresses the limitations of existing range locks. However, this approach comes with its own set of trade-offs. In general, insertion and lookup operations in a linked list are less efficient than tree-like structures. The average time complexity for searching in a linked list is \( O(n) \), whereas it is only \( O(\log n) \) for skip lists or tree-like structures~\parencite{fomitchev2004lock}.


\section{Comparative Analysis and Trade-offs}

The analysis of range lock implementations reveals distinct trade-offs in terms of performance, scalability, and complexity. The interval tree-based range lock in the Linux kernel, despite its fairness and simplicity, suffers from a significant bottleneck due to its reliance on a single spinlock, which can severely limit performance under high contention. The skip list-based range lock offers improved efficiency and dynamic management of address ranges, leveraging the lightweight and fast nature of skip lists. However, it still retains the single contention point, similar to the interval tree approach, which hinders its scalability under heavy concurrent access. The concurrent linked list-based range lock by Kogan et al. presents a lock-free mechanism that significantly enhances scalability by eliminating blocking behavior. Nonetheless, it introduces inefficiencies in insertion and lookup operations, as linked lists inherently have a higher average time complexity compared to skip lists and tree-like structures.

\section{Future Directions}

Future research on range-locking mechanisms should focus on hybrid approaches that combine the advantages of different data structures to mitigate their individual weaknesses. For instance, exploring the integration of lock-free principles with more efficient data structures, such as combining skip lists with lock-free operations, could enhance both performance and scalability. Additionally, investigating adaptive range-locking mechanisms that dynamically switch between different data structures based on the current contention level and workload characteristics could provide optimized performance across varying conditions. Finally, further studies on distributed range-locking techniques could extend these mechanisms to multi-node environments, addressing the growing need for scalable synchronization in distributed systems. By advancing these directions, researchers can develop more robust and efficient range-locking solutions that cater to the demands of modern computing environments.