\chapter{Related Work}\label{chapter:relatedwork}

\section{Coarse-Grained Range Lock}

\subsection{Tree-Based Range Lock}\label{subsec:treebased}

Several works have explored coarse-grained range-locking mechanisms. Jan Kara introduced a range-locking mechanism for the Linux kernel~\parencite{linuxRangeLockImpl2013}, which utilizes a range tree (specifically a red-black tree) to manage range locks and employs a spinlock for synchronization. Each lock is represented as a node in the tree. Similarly, Kim et al. adopted a comparable range-locking mechanism in their work on pNOVA~\parencite{kim2019pnova}, a variant of the NOVA file system that uses range-based reader-writer locks to enable parallel I/O within a single shared file.

When a thread requests a range lock, it first acquires a spinlock, then traverses the tree to determine the number of locks intersecting with the requested range. Afterward, the thread inserts a node describing its range into the tree and releases the spinlock. If no intersecting locks are found, the thread can proceed with accessing the critical section. If intersecting locks are detected, the thread waits until those locks are released and the number of intersecting locks drops to zero. Upon completing its operation, the thread re-acquires the spinlock, removes its node from the tree, updates the count of overlapping locks, and releases the spinlock. This method ensures that each range is locked only after all previous conflicting range locks have been released, thereby achieving fairness and avoiding livelocks.

\subsubsection*{Drawbacks}

One significant observation is that the coarse-grained spinlock of an interval tree can severely hinder parallelism, as the spinlock effectively serializes all incoming lock and unlock requests. Under heavy concurrent access, this serialization easily becomes a contention point, limiting the system's performance.

Consider three exclusive lock requests for the ranges \( A = [1..3], \; B = [2..7], \; \text{and} \; C = [4..5] \), arriving in that order. While A holds the lock, B is blocked because it overlaps with A, and C is blocked behind B. However, in practice, C does not overlap with A and could proceed without waiting. This unnecessary blocking reduces the overall efficiency and concurrency of the system.

\subsection{List-Based Range Lock}\label{subsec:listbase}

Song et al.~\parencite{song2013parallelizing} introduced a dynamic range-locking design to enhance the implementation of the Linux kernel. Their range lock uses a skip list~\parencite{pugh1990skip} to dynamically manage the address ranges that are currently locked.

When a thread requests a specific range \texttt{[start, start+len)}, the range lock searches the skip list. If an existing or overlapping range is found, it indicates that another thread is currently modifying that range, requiring the requesting thread to wait and then retry. If no overlapping range is found, the requested range is added to the skip list, signifying that the lock has been acquired. Releasing a range involves deleting the corresponding range from the skip list.

Compared to the interval tree, the skip list is more lightweight and efficient, allowing for quicker searches of overlapping ranges.

\subsubsection*{Drawbacks}

Similar to the tree-based range lock, contention remains an issue with this approach. Additionally, it unnecessarily blocks non-conflicting requests, further reducing system efficiency and limiting concurrency.

\section{Fine-Grained Range Lock}

\subsection{List-Based Range Lock}

Kogan et al.~\parencite{kogan2020scalable} introduced a range lock based on a concurrent linked list, where each node represents an acquired range. This design aims to provide a lock-free mechanism, addressing critical shortcomings of previous range-locking implementations. In a lock-free system, processes can proceed without being blocked by locks held by other processes, thereby improving performance and scalability.

The proposed method involves inserting acquired ranges into a linked list sorted by their starting points, ensuring that only one range from a group of overlapping ranges can be inserted using an atomic \texttt{compare-and-swap} (CAS) operation. A significant difference in this method compared to previous ones is that each node has two statuses: \texttt{marked} (logically deleted) or \texttt{unmarked} (present).

When a thread wants to acquire a range, it iterates through the skip list. If it encounters a marked node, it removes it using CAS and continues to iterate. If the current node protects a range that overlaps, the thread waits until that node is deleted. Otherwise, a node is inserted into the list, signaling that the range is acquired. To release a range, the thread marks the node as deleted.

\subsubsection*{Drawbacks}

\textbf{Linked List Inefficiency:} While this design implements a lock-free mechanism that effectively addresses the limitations of existing range locks, it comes with its own set of trade-offs. In general, insertion and lookup operations in a linked list are less efficient than in tree-like structures. The average time complexity for searching in a linked list is \( O(n) \), whereas it is only \( O(\log n) \) for skip lists or tree-like structures~\parencite{fomitchev2004lock}. Our evaluation will demonstrate that this inefficiency becomes particularly pronounced when handling multiple overlapping ranges within the list.

\subsection{Bitmap Range Lock}

In addition to the tree-based range locking method discussed in \ref{subsec:treebased}, Kim et al. proposed a lock-free range locking mechanism, which they claim offers enhanced efficiency compared to interval tree-based locks. 
This approach involves dividing a file into segments, each managed by a 32-bit variable that functions as a reader-writer lock. The most significant bit represents the writer lock status (1 for locked, 0 for unlocked), while the remaining 31 bits count the number of active readers. 
The mechanism utilizes hardware-supported atomic operations to ensure that writer locks can only be set when no other locks are active and that reader locks are granted as long as no writer lock is present. 
Unlocking is achieved by clearing the writer lock bit and decrementing the reader counter. 

Although this method provides finer-grained locking with reduced overhead compared to interval tree-based locks, it is tailored to handle both reader and writer modes and depends on specific memory size constraints. 
Since our research focuses exclusively on the exclusive mode and does not address the reader-writer combination, we have chosen not to consider this approach as a variant in our project.

